\documentclass[conference]{IEEEtran}
\usepackage[utf8]{inputenc}
\usepackage[spanish]{babel}
\usepackage{graphicx}
\usepackage{amsmath}
\usepackage{cite}
\usepackage{hyperref}
\usepackage{float}

\title{\textbf{Informe de Laboratorio 0: Equipos de medición en alta frecuencia}} 

\author{
    \IEEEauthorblockN{Felipe Diaz Gordillo, Nelcy Julieth Reyes Sanchez, aaa} 
    \IEEEauthorblockA{
        Facultad de Ingeniería Eléctrica y Electrónica\\
        Universidad Nacional de Colombia \\
        Bogotá, Colombia \\
        fdiazgo, nreyess, aaaa@unal.edu.co}
}

\begin{document}
\maketitle

\begin{abstract}
En esta práctica se estudió el efecto de carga en mediciones de alta frecuencia mediante un barrido en frecuencia aplicado a un circuito de prueba. Se realizaron mediciones utilizando un multímetro digital y un osciloscopio, con el fin de analizar cómo las características internas de los instrumentos influyen en la señal medida. El experimento permitió evidenciar la importancia del ancho de banda y de la impedancia de entrada de los equipos de medición, destacando su impacto en la precisión y confiabilidad de las mediciones a diferentes frecuencias.
\end{abstract}

\begin{IEEEkeywords}
Osciloscopio, Multímetro, Respuesta en frecuencia, Mediciones eléctricas.
\end{IEEEkeywords}

\section{Introducción}
En esta sección, describe el contexto del experimento, los objetivos y cualquier información relevante de antecedentes. Explica por qué el experimento es importante y cuáles son las preguntas o hipótesis principales.

\section{Marco Teórico}
Incluye las bases teóricas necesarias para entender el experimento. Proporciona ecuaciones clave, explicaciones y citas relevantes de la literatura.

\section{Metodología}
Los materiales que se utilizarón para esta practica fueron:
\begin{itemize}
    \item Multimetro Digital (MMD)
    \item Osciloscopio de doble canal
    \item Sondas comunes y sondas atenuadoras
    \item Generador de señales
    \item Resistencias
\end{itemize}

\subsection{Procedimiento}
Antes de realizar mediciones en el laboratorio se realizarón una serie de simulaciones con el objetivo de saber cuales eran los resultados ideales o esperados, esto para poder hacer una comparación relevante de los datos, ademas de entender los fenomenos que deberian suceder. Fue importante hacer una investigacion sobre los elementos de medición que se usarian en la practica, como el osciloscopio y el MMD para conocer los valores de impedancia interna que tienen y a que escalas estos afectan a las mediciones; se incluyo en la investigación la resolucion de los dispositivos a diferentes frecuencias.

Para hacer las medidas en el laboratorio los pasos a seguir fueron:
\begin{enumerate}
    \item Hacer el montaje del circutio que se trabajaria para realizar mediciones, marcando claramente cada uno de los nodos. 
    \item Se realiza la conexion del generador de señales y se configura a una frecuencia de 100Hz.
    \item Se hace la medición de voltaje en cada uno de los nodos utilizando tanto el osciloscopio como el multimetro.
    \item Se hace un barrio de frecuencias desde 1Hz hasta 30MHz, para cada frecuencia elegida se mide cada uno de los nodos y se registran los datos en una tabla.
\end{enumerate}

Posterior al laboratorio se construyeron las graficas \ref{fig:grafica_osci} y \ref{fig:grafica_mmd} que se pueden observar en \ref{sec:Res}, se hizo el analisis de resultados y las comparaciones pertienetes entre los datos de simulacion previos al laboratorio y los datos tomados en la practica.



\section{Resultados}
\label{sec:Res}
Los datos obtenidos con cada uno de los dispositivos de medición se muestran en las tablas a continuación.

\begin{table}[H]
\centering
\caption{Mediciones realizadas con el Osciloscopio}
\label{tab:osciloscopio}
\begin{tabular}{|c|c|c|c|}
\hline
\textbf{Frecuencia} & \textbf{Nodo A} & \textbf{Nodo B} & \textbf{Nodo C} \\ \hline
100 Hz (Ref) & 364 & 182 & 124 \\ \hline
10 Hz        & 295 & 145 & 110 \\ \hline
30 Hz        & 350 & 174 & 120 \\ \hline
70 Hz        & 358 & 178 & 121 \\ \hline
200 Hz       & 360 & 179 & 121 \\ \hline
500 Hz       & 360 & 179 & 121 \\ \hline
1 kHz        & 360 & 179 & 119 \\ \hline
2 kHz        & 360 & 179 & 113 \\ \hline
5 kHz        & 360 & 180 & 85.8 \\ \hline
10 kHz       & 354 & 178 & 55.8 \\ \hline
50 kHz       & 355 & 176 & 14 \\ \hline
100 kHz      & 355 & 170 & 9 \\ \hline
500 kHz      & 354 & 102 & 6.8 \\ \hline
1 MHz        & 354 & 59  & 10.9 \\ \hline
5 MHz        & 348 & 14  & 7.11 \\ \hline
15 MHz       & 231 & 7.1 & 6 \\ \hline
25 MHz       & 130 & 5.1 & 6 \\ \hline
\end{tabular}
\end{table}

\begin{figure}[H]
    \centering
    \includegraphics[width=0.8\linewidth]{medicionesOSCI.png}
    \caption{Gráfica mediciones osciloscopio}
    \label{fig:grafica_osci}
\end{figure}

\begin{table}[H]
\centering
\caption{Mediciones realizadas con el Multímetro (MMD)}
\label{tab:multimetro}
\begin{tabular}{|c|c|c|c|}
\hline
\textbf{Frecuencia} & \textbf{Nodo A} & \textbf{Nodo B} & \textbf{Nodo C} \\ \hline
100 Hz (Ref) & 357 & 178 & 182 \\ \hline
10 Hz        & 345 & 172 & 177 \\ \hline
30 Hz        & 357 & 180 & 250 \\ \hline
70 Hz        & 357 & 180 & 177 \\ \hline
200 Hz       & 357 & 178 & 177 \\ \hline
500 Hz       & 357 & 178 & 248 \\ \hline
1 kHz        & 354 & 177 & 267 \\ \hline
2 kHz        & 344 & 173 & 272 \\ \hline
5 kHz        & 302 & 152 & 227 \\ \hline
10 kHz       & 246 & 124 & 81 \\ \hline
50 kHz       & 160 & 44  & 69 \\ \hline
100 kHz      & 70  & 10  & 62 \\ \hline
500 kHz      & 6   & 7   & 173 \\ \hline
1 MHz        & 6   & 6   & 67 \\ \hline
5 MHz        & 6   & 6   & 70 \\ \hline
15 MHz       & 6   & 6   & 72 \\ \hline
25 MHz       & 6   & 6   & 72 \\ \hline
\end{tabular}
\end{table}

\begin{figure}[H]
    \centering
    \includegraphics[width=0.8\linewidth]{medicionesMMD.png}
    \caption{Gráfica mediciones multimetro digital}
    \label{fig:grafica_mmd}
\end{figure}

\section{Discusión}
Se observa que en las mediciones realizadas con el multímetro digital (MMD), a partir de los $100$ kHz, los valores de voltaje decaen abruptamente hasta estabilizarse en un valor constante de $6$ mV para frecuencias superiores a $500$ kHz. Este fenómeno no representa la señal real del circuito (como se comprueba con el osciloscopio), sino que evidencia la limitación en el ancho de banda del multímetro. Al superar su frecuencia de corte operacional, el dispositivo es incapaz de rectificar la señal de AC, mostrando únicamente su suelo de ruido o tensión residual de offset.



\section{Conclusiones}
Las mediciones son consistentes en bajas frecuencias, validando el uso de ambos equipos para señales de audio, pero restringiendo el MMD para aplicaciones de alta frecuencia.

\bibliographystyle{IEEEtran}
% \bibliography{referencias} % Descomenta esto solo si tienes un archivo referencias.bib
\end{document}
