\documentclass[conference]{IEEEtran}
\usepackage[utf8]{inputenc} 
\usepackage{graphicx} 
\usepackage{amsmath}
\usepackage{cite}
\usepackage{hyperref} 

\title{\textbf{Informe de Laboratorio 0: Equipos de medición en alta frecuencia}} 

\author{
    \IEEEauthorblockN{Felipe Diaz Gordillo, aaa, aaa} 
    \IEEEauthorblockA{
        Facultad de Ingeniería Eléctrica y Electrónica\\
        Universidad Nacional de Colombia \\
        Bogotá, Colombia \\
        fdiazgo, aaaa, aaaa@unal.edu.co}
}

\begin{document}
\maketitle

\begin{abstract}
Este informe presenta el análisis comparativo de mediciones de voltaje en diversos nodos de un circuito utilizando un osciloscopio y un multímetro digital (MMD) a través de un amplio rango de frecuencias. Se incluye la explicación del comportamiento de diversos componentes utilizados en las mediciones de laboratorio y como estos pueden afectar el comportamiento de los circutios.
\end{abstract}

\begin{IEEEkeywords}
Osciloscopio, Multímetro, Respuesta en frecuencia, Mediciones eléctricas, Sondas.
\end{IEEEkeywords}

\section{Introducción}
En esta sección, describe el contexto del experimento, los objetivos y cualquier información relevante de antecedentes. Explica por qué el experimento es importante y cuáles son las preguntas o hipótesis principales.

\section{Marco Teórico}
Incluye las bases teóricas necesarias para entender el experimento. Proporciona ecuaciones clave, explicaciones y citas relevantes de la literatura.

\section{Metodología}
Describe los materiales y métodos empleados en el experimento de forma clara y detallada. Incluye:
- Lista de materiales utilizados.
- Procedimientos experimentales paso a paso.

\section{Resultados}
Los datos obtenidos con cada uno de los dispositivos de medición se muestran en las tablas a continuación.
% Tabla 1: Osciloscopio
\begin{table}[h]
\centering

\label{tab:osciloscopio}
\begin{tabular}{|c|c|c|c|}
\hline
\textbf{Frecuencia} & \textbf{Nodo A} & \textbf{Nodo B} & \textbf{Nodo C} \\ \hline
100 Hz (Ref) & 364 & 182 & 124 \\ \hline
10 Hz        & 295 & 145 & 110 \\ \hline
30 Hz        & 350 & 174 & 120 \\ \hline
70 Hz        & 358 & 178 & 121 \\ \hline
200 Hz       & 360 & 179 & 121 \\ \hline
500 Hz       & 360 & 179 & 121 \\ \hline
1 kHz        & 360 & 179 & 119 \\ \hline
2 kHz        & 360 & 179 & 113 \\ \hline
5 kHz        & 360 & 180 & 85.8 \\ \hline
10 kHz       & 354 & 178 & 55.8 \\ \hline
50 kHz       & 355 & 176 & 14 \\ \hline
100 kHz      & 355 & 170 & 9 \\ \hline
500 kHz      & 354 & 102 & 6.8 \\ \hline
1 MHz        & 354 & 59  & 10.9 \\ \hline
5 MHz        & 348 & 14  & 7.11 \\ \hline
15 MHz       & 231 & 7.1 & 6 \\ \hline
25 MHz       & 130 & 5.1 & 6 \\ \hline
\end{tabular}
\end{table}
\caption{Mediciones realizadas con el Osciloscopio}
\begin{figure}[H]
    \centering
    \includegraphics[width=0.5\linewidth]{medicionesOSCI.png}
    \caption{Gráfica mediciones osciloscopio}
    \label{fig:placeholder}
\end{figure}


% Tabla 2: Multímetro
\begin{table}[h]
\centering
\label{tab:multimetro}
\begin{tabular}{|c|c|c|c|}
\hline
\textbf{Frecuencia} & \textbf{Nodo A} & \textbf{Nodo B} & \textbf{Nodo C} \\ \hline
100 Hz (Ref) & 357 & 178 & 182 \\ \hline
10 Hz        & 345 & 172 & 177 \\ \hline
30 Hz        & 357 & 180 & 250 \\ \hline
70 Hz        & 357 & 180 & 177 \\ \hline
200 Hz       & 357 & 178 & 177 \\ \hline
500 Hz       & 357 & 178 & 248 \\ \hline
1 kHz        & 354 & 177 & 267 \\ \hline
2 kHz        & 344 & 173 & 272 \\ \hline
5 kHz        & 302 & 152 & 227 \\ \hline
10 kHz       & 246 & 124 & 81 \\ \hline
50 kHz       & 160 & 44  & 69 \\ \hline
100 kHz      & 70  & 10  & 62 \\ \hline
500 kHz      & 6   & 7   & 173 \\ \hline
1 MHz        & 6   & 6   & 67 \\ \hline
5 MHz        & 6   & 6   & 70 \\ \hline
15 MHz       & 6   & 6   & 72 \\ \hline
25 MHz       & 6   & 6   & 72 \\ \hline
\end{tabular}
\end{table}
\caption{Mediciones realizadas con el Multímetro (MMD)}
\begin{figure}[H]
    \centering
    \includegraphics[width=0.5\linewidth]{medidasMMD.png}
    \caption{Gráfica medidas multimetro digital}
    \label{fig:placeholder}
\end{figure}

\section{Discusión}
Se observa que en las mediciones realizadas con el multímetro digital (MMD), a partir de los $100$ kHz, los valores de voltaje decaen abruptamente hasta estabilizarse en un valor constante de $6$ mV para frecuencias superiores a $500$ kHz. Este fenómeno no representa la señal real del circuito (como se comprueba con el osciloscopio), sino que evidencia la limitación en el ancho de banda del multímetro. Al superar su frecuencia de corte operacional, el dispositivo es incapaz de rectificar la señal de AC, mostrando únicamente su suelo de ruido o tensión residual de offset.



\section{Conclusiones}
Las mediciones son consistentes en bajas frecuencias, validando el uso de ambos equipos para señales de audio, pero restringiendo el MMD para aplicaciones de alta frecuencia.

\bibliographystyle{IEEEtran}
% \bibliography{referencias} % Descomenta esto solo si tienes un archivo referencias.bib
\end{document}
