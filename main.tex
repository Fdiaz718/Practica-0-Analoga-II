\documentclass[conference]{IEEEtran}
\usepackage[utf8]{inputenc} 
\usepackage{graphicx} 
\usepackage{amsmath}
\usepackage{cite}
\usepackage{hyperref} 
% Título y autores
\title{\textbf{Plantilla}} 
\author{
    \IEEEauthorblockN{Felipe Diaz Gordillo} 
    \IEEEauthorblockA{
        Facultad de Ingeniera Eléctrica y Electrónica\\
        Universidad Nacional de Colombia \\
        22 de Enero de 2026 \\
        Correo Electrónico: \text{fdiazgo, aaa, aaa - @unal.edu.co}}
}



\begin{document}
\maketitle
\begin{abstract}
El resumen debe proporcionar una visión general breve de los objetivos, el enfoque, los resultados clave y las conclusiones del informe. Escribe entre 150 y 250 palabras.
\end{abstract}
\begin{IEEEkeywords}
PALABRAS CLAVE
\end{IEEEkeywords}





\section{Introducción}
En esta sección, describe el contexto del experimento, los objetivos y cualquier información relevante de antecedentes. Explica por qué el experimento es importante y cuáles son las preguntas o hipótesis principales.

\section{Marco Teórico}
Incluye las bases teóricas necesarias para entender el experimento. Proporciona ecuaciones clave, explicaciones y citas relevantes de la literatura.

\section{Metodología}
Describe los materiales y métodos empleados en el experimento de forma clara y detallada. Incluye:
- Lista de materiales utilizados.
- Procedimientos experimentales paso a paso.

\section{Resultados}
Los datos que se obtuvieron con cada uno de los dispositivos de medición se muestran en las tablas a continuación:
\begin{table}[h]
\centering
\label{tab:osciloscopio}
\begin{tabular}{|c|c|c|c|}
\hline
\textbf{Frecuencia} & \textbf{Nodo A (mV)} & \textbf{Nodo B (mV)} & \textbf{Nodo C (mV)} \\ \hline
100 Hz (Ref)        & 364                  & 182                  & 124                  \\ \hline
10 Hz               & 295                  & 145                  & 110                  \\ \hline
30 Hz               & 350                  & 174                  & 120                  \\ \hline
70 Hz               & 358                  & 178                  & 121                  \\ \hline
200 Hz              & 360                  & 179                  & 121                  \\ \hline
500 Hz              & 360                  & 179                  & 121                  \\ \hline
1 kHz               & 360                  & 179                  & 119                  \\ \hline
2 kHz               & 360                  & 179                  & 113                  \\ \hline
5 kHz               & 360                  & 180                  & 85.8                 \\ \hline
10 kHz              & 354                  & 178                  & 55.8                 \\ \hline
50 kHz              & 355                  & 176                  & 14                   \\ \hline
100 kHz             & 355                  & 170                  & 9                    \\ \hline
500 kHz             & 354                  & 102                  & 6.8                  \\ \hline
1 MHz               & 354                  & 59                   & 10.9                 \\ \hline
5 MHz               & 348                  & 14                   & 7.11                 \\ \hline
15 MHz              & 231                  & 7.1                  & 6                    \\ \hline
25 MHz              & 130                  & 5.1                  & 6                    \\ \hline
\caption{Mediciones realizadas con el Osciloscopio}
\end{tabular}
\end{table}


\begin{table}[h]
\centering
\label{tab:multimetro}
\begin{tabular}{|c|c|c|c|}
\hline
\textbf{Frecuencia} & \textbf{Nodo A (mV)} & \textbf{Nodo B (mV)} & \textbf{Nodo C (mV)} \\ \hline
100 Hz (Ref)        & 357                  & 178                  & 182                  \\ \hline
10 Hz               & 345                  & 172                  & 177                  \\ \hline
30 Hz               & 357                  & 180                  & 250                  \\ \hline
70 Hz               & 357                  & 180                  & 177                  \\ \hline
200 Hz              & 357                  & 178                  & 177                  \\ \hline
500 Hz              & 357                  & 178                  & 248                  \\ \hline
1 kHz               & 354                  & 177                  & 267                  \\ \hline
2 kHz               & 344                  & 173                  & 272                  \\ \hline
5 kHz               & 302                  & 152                  & 227                  \\ \hline
10 kHz              & 246                  & 124                  & 81                   \\ \hline
50 kHz              & 160                  & 44                   & 69                   \\ \hline
100 kHz             & 70                   & 10                   & 62                   \\ \hline
500 kHz             & 6                    & 7                    & 173                  \\ \hline
1 MHz               & 6                    & 6                    & 67                   \\ \hline
5 MHz               & 6                    & 6                    & 70                   \\ \hline
15 MHz              & 6                    & 6                    & 72                   \\ \hline
25 MHz              & 6                    & 6                    & 72                   \\ \hline
\caption{Mediciones realizadas con el Multímetro (MMD) (Valores en mV)}
\end{tabular}
\end{table}





\section{Discusión}
Analiza los resultados obtenidos en el contexto de los objetivos del experimento. Discute las posibles fuentes de error, limitaciones y cómo los resultados se comparan con expectativas o teorías previas.

\section{Conclusiones}
Resume los puntos clave del experimento, sus hallazgos y su relevancia. Incluye recomendaciones para investigaciones futuras, si corresponde.


\bibliographystyle{IEEEtran}
\bibliography{referencias} %citar en texto
\end{document}
