\documentclass[conference]{IEEEtran}
\usepackage[utf8]{inputenc}
\usepackage[spanish]{babel}
\usepackage{graphicx}
\usepackage{amsmath}
\usepackage{cite}
\usepackage{hyperref}
\usepackage{float}

\title{\textbf{Informe de Laboratorio 0: Equipos de medición en alta frecuencia}} 

\author{
    \IEEEauthorblockN{Felipe Diaz Gordillo, Nelcy Julieth Reyes Sanchez, Gabriel Jaime Herrera Alfaro} 
    \IEEEauthorblockA{
        Facultad de Ingeniería Eléctrica y Electrónica\\
        Universidad Nacional de Colombia \\
        Bogotá, Colombia \\
        fdiazgo, nreyess, gaherreraa, @unal.edu.co}
}

\begin{document}
\maketitle

\begin{abstract}
En esta práctica se estudió el efecto de carga en mediciones de alta frecuencia mediante un barrido en frecuencia aplicado a al circuito presentado en la guía. Se realizaron mediciones utilizando un multímetro digital y un osciloscopio, con el fin de analizar cómo las características internas de los instrumentos influyen en la señal medida. El experimento permitió evidenciar la importancia del ancho de banda y de la impedancia de entrada de los equipos de medición, destacando su impacto en la precisión y confiabilidad de las mediciones a diferentes frecuencias.
\end{abstract}

\begin{IEEEkeywords}
Osciloscopio, Multímetro, Frecuencia, Mediciones eléctricas, Efecto de Carga.
\end{IEEEkeywords}

\section{Introducción}
La medición de señales en alta frecuencia presenta desafíos importantes debido a la interacción entre el instrumento y el circuito bajo prueba. Uno de los fenómenos más relevantes es el efecto de carga, mediante el cual las características internas del equipo de medición alteran la respuesta del circuito, generando errores en los valores observados. Este efecto se vuelve especialmente significativo cuando se emplean instrumentos con limitaciones en su ancho de banda o impedancias no ideales.

En el presente informe se estudia experimentalmente el efecto de carga mediante un barrido en frecuencia aplicado al circuito de la guía. Se analizan y comparan las mediciones obtenidas con un multímetro digital y un osciloscopio, con el objetivo de identificar las limitaciones de cada equipo y su impacto en la exactitud de la medición.



\section{Marco Teórico}
En el proceso de medición de los circuitos electrónicos, los equipos de medición no son ideales y siempre presentan una impedancia de entrada que puede afectar el funcionamiento del circuito que se desea medir. Por ejemplo, cuando se conecta un instrumento a un nodo, su impedancia se suma al circuito y puede modificar el valor real de la señal, a este fenómeno se le conoce como como efecto de carga. Este efecto es más notable cuando se miden nodos de alta impedancia o cuando se trabaja a frecuencias elevadas, ya que las mediciones dejan de representar fielmente el comportamiento del circuito.

Los instrumentos como el multímetro digital y el osciloscopio pueden modelarse mediante una resistencia en paralelo con una capacitancia. A bajas frecuencias, el efecto de la capacitancia es pequeño y la medición puede considerarse como confiable, pero a altas frecuencias la impedancia disminuye y la señal medida se ve afectada. Por esta razón, el uso de sondas atenuadoras y la simulación con modelos realistas de los componentes del circuito permite identificar los rangos de frecuencia en los que el efecto de carga es despreciable y mejorar la comparación entre los resultados teóricos, simulados y experimentales.

\section{Trabajo previo}

A partir de la hoja de datos de los dispositivos empleados en la práctica de laboratorio, se obtuvieron las siguientes especificaciones técnicas:

\subsection{Sonda atenuadora:}
Es el cable con una punta que se usa para conectar el circuito al osciloscopio. Ayuda a que la señal que se quiere medir llegue clara y sin interferencias, especialmente cuando es una señal que cambia lentamente. La que se encuentran en el laboratorio es una P6100 BNC, y operando en su posición de atenuación X1. De acuerdo con su ficha técnica, sus parámetros son:

\begin{itemize}
    \item Impedancia resistiva: 1 MΩ.
    \item Capacidad parásita: 90pF.
\end{itemize}

\subsubsection{Osciloscopio:}
Es una pantalla que representa cómo cambia el voltaje de una señal con el paso del tiempo. Básicamente, nos permite "ver" la electricidad.
El modelo que usamos fue el \textit{Tektronix TBS 1102B-EDU}. Tiene una función muy importante llamada "disparo" o "trigger", que sirve para congelar la imagen en la pantalla y poder ver la señal quieta; si no, la onda se movería todo el tiempo y no se podria analizar detalladamente. Sus características de entrada (lo que el osciloscopio "ofrece" al circuito que se esta midiendo) son:
\begin{itemize}
    \item Resistencia: 1 MΩ.
    \item Capacitancia: 20pF.
\end{itemize}

\subsubsection{Generador de señales:}
Es un aparato que fabrica ondas eléctricas (como senos, cuadradas, triangulares) para poder inyectarlas en un circuito y probar cómo se comporta. Es como un "instrumento de prueba" que se usa para simular diferentes condiciones. Su característica principal es:
\begin{itemize}
    \item Resistencia interna de salida: 50Ω.
\end{itemize}


\section{Metodología}

Los materiales que se utilizarón para esta practica fueron:
\begin{itemize}
    \item Multimetro Digital (MMD)
    \item Osciloscopio de doble canal
    \item Sondas comunes y sondas atenuadoras
    \item Generador de señales
    \item Resistencias
\end{itemize}

\subsection{Procedimiento}
Antes de realizar mediciones en el laboratorio se realizarón una serie de simulaciones con el objetivo de saber cuales eran los resultados ideales o esperados, esto para poder hacer una comparación relevante de los datos, ademas de entender los fenomenos que deberian suceder. Fue importante hacer una investigacion sobre los elementos de medición que se usarian en la practica, como el osciloscopio y el MMD para conocer los valores de impedancia interna que tienen y a que escalas estos afectan a las mediciones; se incluyo en la investigación la resolucion de los dispositivos a diferentes frecuencias.

Para hacer las medidas en el laboratorio los pasos a seguir fueron:
\begin{enumerate}
    \item Hacer el montaje del circutio que se trabajaria para realizar mediciones, marcando claramente cada uno de los nodos. 
    \item Se realiza la conexion del generador de señales y se configura a una frecuencia de 100Hz.
    \item Se hace la medición de voltaje en cada uno de los nodos utilizando tanto el osciloscopio como el multimetro.
    \item Se hace un barrio de frecuencias desde 1Hz hasta 30MHz, para cada frecuencia elegida se mide cada uno de los nodos y se registran los datos en una tabla.
\end{enumerate}

Posterior al laboratorio se construyeron las graficas \ref{fig:grafica_osci} y \ref{fig:grafica_mmd} que se pueden observar en \ref{sec:Res}, se hizo el analisis de resultados y las comparaciones pertienetes entre los datos de simulacion previos al laboratorio y los datos tomados en la practica.



\section{Resultados}
\label{sec:Res}
Los datos obtenidos con cada uno de los dispositivos de medición se muestran en las tablas a continuación.

\begin{table}[H]
\centering
\caption{Mediciones realizadas con el Osciloscopio}
\label{tab:osciloscopio}
\begin{tabular}{|c|c|c|c|}
\hline
\textbf{Frecuencia} & \textbf{Nodo A} & \textbf{Nodo B} & \textbf{Nodo C} \\ \hline
100 Hz (Ref) & 364 & 182 & 124 \\ \hline
10 Hz        & 295 & 145 & 110 \\ \hline
30 Hz        & 350 & 174 & 120 \\ \hline
70 Hz        & 358 & 178 & 121 \\ \hline
200 Hz       & 360 & 179 & 121 \\ \hline
500 Hz       & 360 & 179 & 121 \\ \hline
1 kHz        & 360 & 179 & 119 \\ \hline
2 kHz        & 360 & 179 & 113 \\ \hline
5 kHz        & 360 & 180 & 85.8 \\ \hline
10 kHz       & 354 & 178 & 55.8 \\ \hline
50 kHz       & 355 & 176 & 14 \\ \hline
100 kHz      & 355 & 170 & 9 \\ \hline
500 kHz      & 354 & 102 & 6.8 \\ \hline
1 MHz        & 354 & 59  & 10.9 \\ \hline
5 MHz        & 348 & 14  & 7.11 \\ \hline
15 MHz       & 231 & 7.1 & 6 \\ \hline
25 MHz       & 130 & 5.1 & 6 \\ \hline
\end{tabular}
\end{table}

\begin{figure}[H]
    \centering
    \includegraphics[width=0.8\linewidth]{Picture1.png}
    \caption{Gráfica mediciones osciloscopio}
    \label{fig:grafica_osci}
\end{figure}

\begin{table}[H]
\centering
\caption{Mediciones realizadas con el Multímetro (MMD)}
\label{tab:multimetro}
\begin{tabular}{|c|c|c|c|}
\hline
\textbf{Frecuencia} & \textbf{Nodo A} & \textbf{Nodo B} & \textbf{Nodo C} \\ \hline
100 Hz (Ref) & 357 & 178 & 182 \\ \hline
10 Hz        & 345 & 172 & 177 \\ \hline
30 Hz        & 357 & 180 & 250 \\ \hline
70 Hz        & 357 & 180 & 177 \\ \hline
200 Hz       & 357 & 178 & 177 \\ \hline
500 Hz       & 357 & 178 & 248 \\ \hline
1 kHz        & 354 & 177 & 267 \\ \hline
2 kHz        & 344 & 173 & 272 \\ \hline
5 kHz        & 302 & 152 & 227 \\ \hline
10 kHz       & 246 & 124 & 81 \\ \hline
50 kHz       & 160 & 44  & 69 \\ \hline
100 kHz      & 70  & 10  & 62 \\ \hline
500 kHz      & 6   & 7   & 173 \\ \hline
1 MHz        & 6   & 6   & 67 \\ \hline
5 MHz        & 6   & 6   & 70 \\ \hline
15 MHz       & 6   & 6   & 72 \\ \hline
25 MHz       & 6   & 6   & 72 \\ \hline
\end{tabular}
\end{table}

\begin{figure}[H]
    \centering
    \includegraphics[width=0.8\linewidth]{Picture2.png}
    \caption{Gráfica mediciones multimetro digital}
    \label{fig:grafica_mmd}
\end{figure}

\section{Discusión}

Se observa que en las mediciones realizadas con el multímetro digital (MMD), a partir de los 100 kHz, los valores de voltaje decaen abruptamente hasta estabilizarse en un valor constante de $6$ mV para frecuencias superiores a $500$ kHz. Este fenómeno no representa la señal real del circuito (como se comprueba con el osciloscopio), sino que evidencia la limitación en el ancho de banda del multímetro. Al superar su frecuencia de corte operacional, el dispositivo es incapaz de rectificar la señal de AC, mostrando únicamente su suelo de ruido o tensión residual de offset.\\

En el osciloscopio se puede observar un comportamiento mas alineado a la realidad, esto gracias al uso de sondas ateniadas 10x que reducen la carga que el equipo de medición le pone al circuito, evitando que los cables actuen como un filtro y por lo tanto recorten la señal en altas frecuencias. Por eso, mientras el multímetro deja de ver la señal casi por completo, las sondas atenuadas nos permiten seguir midiendo con claridad hasta los 25 MHz con buena confiabilidad.\\

Todo esto es dado por el efecto de carga que se comento anteriormente en el documento, que ocurre cuando el instrumento de medición extrae corriente del circuito y altera el voltaje real del nodo. Como en bajas freceuncias la resistencia del multimetro es relativamente alta a comparación del circuito, no se producen estos efectos de filtro al momento de la medición; pero al aumentar la frecuencia, su impedancia cae y empieza a comportarse como una carga pesada que consume corriente de la señal. Las sondas atenuadas del osciloscopio minimizan este problema, ya que presentan una resistencia mucho mayor 10 (M$\Omega$) y una capacitancia muy pequeña. Sin embargo, en las medidas de 15 MHz y 25 MHz, se nota que incluso con estas sondas el voltaje disminuye; esto sucede porque a frecuencias tan altas, ninguna sonda es totalmente invisible y siempre terminan cargando un poco el circuito, lo que explica por qué los valores medidos son menores a los esperados teóricamente


\section{Conclusiones}
Las mediciones son consistentes en bajas frecuencias, validando el uso de ambos equipos para señales de audio, pero restringiendo el MMD para aplicaciones de alta frecuencia.

\bibliographystyle{IEEEtran}
% \bibliography{referencias} % Descomenta esto solo si tienes un archivo referencias.bib
\end{document}
